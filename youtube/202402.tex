\documentclass {article}
\usepackage {ctex}
\usepackage{amsmath}% For the equation* environment
\usepackage[dvips]{graphicx}
\usepackage[colorlinks=true, urlcolor=blue, linkcolor=red]{hyperref}
\begin{document}
\section{a very interesting differential equation}
\href{https://www.youtube.com/watch?v=tr7NZ71LFKA}{a very interesting differential equation}


Problem  \[ y^{'''} = y^{'}y^{''}\]

suppose
\[
\begin{matrix}
    & y^{'''}  = y^{'}y^{''} \\
     \stackrel {z = y^{'}}{\implies} & z^{''} = z z^{'} \\
     \implies & 2z^{''} = \left(2z\right)z^{'} \\
     \implies & \frac{{d}\left(2z^{'}\right)}{{d}x} = \frac {{d}\left(z^{2}\right)} {{d}x}  \\
     \implies & 2z^{'} = z^{2} + C \\
\end{matrix}
\]


one case.
\[
\begin{matrix}
 & C = 0 \\
\implies & 2 \frac {{d}z} {{d}x} = z ^{2} \\
\implies & \int \frac {2}{z^2} {d}z = \int {d}x \\
\implies & -\frac{2}{z} = x - D  \left(\text{D is constant}\right)\\
\implies & z =  \frac{2}{D -x} \\
\implies & y^{'} = \frac{2}{D -x } \\
\implies & y = E - 2\ln\left(D-x\right) (\text{E is constant}) \\
\end{matrix}
\]

the second case.
\[
\begin{matrix}
 & C > 0 \\
\implies & 2 \frac {{d}z} {{d}x} = z^{2} + C \\
\implies & 2 \int \frac{1}{z^{2} + C} {d}z = \int {d}x \\
\implies & 2\frac {\arctan\left( \frac {z}{\sqrt{C}} \right)} {\sqrt{C}} = x + D \left(\text{D is constant}\right) \\
\implies & \arctan\left(\frac {z} {\sqrt{C} }  \right) = \frac {\sqrt{C}x + \sqrt{C}D}{2} \\
\implies & z = \sqrt{C}\tan\left( \frac {\sqrt{C}x + \sqrt{C}D}{2} \right) \\
\implies & y^{'}= \sqrt{C}\tan\left( \frac {\sqrt{C}x + \sqrt{C}D}{2} \right) \\
\implies & y = 2\sqrt{C} \ln\left(\sec\left(\frac{\sqrt{C}x + \sqrt{C}D}{2}\right)\right) + E \\
\end{matrix}
\]

\section{Easy and Hard Integral}
\href{https://www.youtube.com/watch?v=o4TR62QvQJU}{Easy and Hard Integral}
Solve Problem \\
\[
\begin{matrix}
     & \int_0^{\frac{\pi}{4}} \ln\left(1 + \tan\left(x\right)\right){d}x  \\
\implies &  I = \int_0^{\frac{\pi}{4}} \ln\left(1 + \tan\left(\frac{\pi}{4} - x\right)\right){d}x
  \left( \ \text{for king's rule}\ \int_a^b f(x){d} = \int_a^b f(a+b-x) {d}x \right) \\
\implies & I = \int_0^{\frac{\pi}{4}} \ln\left( 1 + \frac{1 - \tan(x)}{1+ \tan(x)} \right) {d}x \\
\implies & I = \int_0^{\frac{\pi}{4}} \ln\left(\frac {2}{1 + \tan(x)}\right) {d}x \\
\implies & I = \int_0^{\frac{\pi}{4}} \ln2 {d}x - \int_0^{\frac{\pi}{4}} ln\left(1 + \tan(x)\right) {d}x \\
\implies & 2I = \int_0^{\frac{\pi}{4}} \ln2 {d}x \\
\implies & I = \frac{\ln2\pi}{8} \\

\text{Solve Problem} \\
 & \int_0^{\frac{\pi}{4}} \ln\left(1 - \tan\left(x\right)\right){d}x  \\
\implies & I = \int_0^{\frac{\pi}{4}} \ln\left(1 - \tan\left(x\right)\right){d}x \\
\implies & I = \int_0^{\frac{\pi}{4}} \ln\left(\frac {\cos(x) - \sin(x)}{\cos(x)}\right){d}x \\
\implies & I = I_1 - I_2 \\
\implies & I_1 = \int_0^{\frac{\pi}{4}} \ln\left(\cos(x) - \sin(x)\right){d}x \text{ } I_2 = \int_0^{\frac{\pi}{4}} \ln\left(\cos(x)\right){d}x \\
\implies & I_1 = \int_0^{\frac{\pi}{4}} \ln\left(\sqrt{2}\left(\frac{1}{\sqrt{2} }\cos(x) - \frac{1}{\sqrt{2}}\sin(x)\right)\right){d}x \\
\implies & I_1 = \int_0^{\frac{\pi}{4}} \ln(\sqrt{2}){d}x + \int_0^{\frac{\pi}{4}}\ln\left(\sin(\frac{\pi}{4} - x)\right){d}x\\
\implies & I_1 = \ln2 \frac{\pi}{8} + \int_0^{\frac{\pi}{4}}\ln\left(\sin(x)\right){d}x \\
\implies & I = I_1 + I_2 \\
\implies & I = \ln2 \frac{\pi}{8} + \int_0^{\frac{\pi}{4}}\ln\left(\sin(x)\right){d}x - \int_0^{\frac{\pi}{4}} \ln\left(\cos(x)\right){d}x \\
\implies & I = \ln2 \frac{\pi}{8} + \int_0^{\frac{\pi}{4}} \ln(\tan(x)){d}x \\
\implies & P = \int_0^{\frac{\pi}{4}}\ln(\tan(x)){d}x = \int_0^{1} \frac{\ln(y)} {1 + y^2}{d}y \text{ } \left(y = \tan(x)\right) \\
\implies & P = \displaystyle\Sigma_{k=0}^{\infty} \int_0^1(-1)^k \ln(y)y^{2k} {d}y \\
\end{matrix}
\]

\section{A beautifully symmetric double integral resulting in an important constant}
\href{https://www.youtube.com/watch?v=MVgDeJaLcd0}{A beautifully symmetric double integral resulting in an important constant}

Solve Problem \\
\[
    I = \displaystyle\int_0^{\frac{\pi}{2}}\int_0^{\frac{\pi}{2}} \frac{\sin(x) + \sin(y)}{\cos(x) + \cos(y)} {d}x{d}y \\
\]

\[
\begin{matrix}
    & I = I_1 + I_2 \\
    & I_1 = \int_0^{\frac{\pi}{2}}\int_0^{\frac{\pi}{2}} \frac{\sin(x)}{\cos(x)+ \cos(y)}{d}x{d}y \\
    & I_2 = \int_0^{\frac{\pi}{2}}\int_0^{\frac{\pi}{2}} \frac{\sin(y)}{\cos(x)+ \cos(y)}{d}y{d}x \\
\implies & I = 2 * I_1 \\
\implies & I = 2 \int_0^{\frac{\pi}{2}}\int_0^{\frac{\pi}{2}} \frac{\sin(x)}{\cos(x)+ \cos(y)}{d}x{d}y \\
\implies & I = -2 \int_0^{\frac{\pi}{2}} \ln\left(\cos(x) + \cos(y)\right)\Big|_{x=0}^{x=\frac{\pi}{2}}{d}y \\
\implies & I = -2 \int_0^{\frac{\pi}{2}} \ln\left(\cos(y)\right){d}y + 2 \int_0^{\frac{\pi}{2}}\ln\left(1 + \cos(y)\right){d}y \\
\implies & I = K_1 + K_2 \\
\implies & K_1 = 2 \int_0^{\frac{\pi}{2}}\ln\left(1 + \cos(y)\right){d}y \\
\implies & K_2 = -2 \int_0^{\frac{\pi}{2}} \ln\left(\cos(y)\right){d}y \\
\implies & K_1 = 2 \int_0^{\frac{\pi}{2}}\ln\left(2 \cos^2(\frac{y}{2})\right){d}y \text{   } \left( 2\cos^2(\frac{y}{2}) = 1 + \cos(y) \right) \\
\implies & K_1 = 2 \int_0^{\frac{\pi}{2}} \ln2{d}y + 4 \int_0^{\frac{\pi}{2}}\ln\left(\cos(\frac{y}{2})\right){d}y \\
\implies & K_1 = \pi\ln2  + 4 * Q_1 \text{   } Q_1 = \int_0^{\frac{\pi}{2}}\ln\left(\cos(\frac{y}{2})\right){d}y \\
\implies & Q_1 = 2\int_0^{\frac{\pi}{4}}\ln\left(\cos(u)\right){d}u  \text{      } u = \frac{y}{2} \\
\implies & \ln\left(\cos(u)\right) = -\ln2 + \displaystyle\sum_{k \ge 1}^{\infty} \frac{(-1)^{k+1}\cos(2ku)}{k} \text{ this is the taylor's expansion for }\ln(u) \\
\implies & Q_1 = \int_0^{\frac{\pi}{4}}\left(-\ln2 + \displaystyle\sum_{k \ge 1}^{\infty} \frac{(-1)^{k+1}\cos(2ku)}{k}\right){d}u \\
\implies & Q_1 = -\ln2 \frac{\pi}{4} + \displaystyle\sum_{N \ge 1}^{\infty}  \frac{(-1)^N}{(2N-1)^2} \text{ because}   \int_0^{\frac{\pi}{4}}\displaystyle\sum_{k \ge 1}^{\infty}\frac{(-1)^{k+1}\cos(2ku)}{k} = \displaystyle\sum_{k \ge 1}^{\infty}\frac{(-1)^{(k+2)}\sin(2ku)}{2k^2}\Big|_{u=0}^{u=\frac{\pi}{4}} \\
\end{matrix}
\]

\section{The INSANE story of the infinite stack integral}
\href{https://www.youtube.com/watch?v=R1ssF4iS3mk}{The INSANE story of the infinite stack integral}

Solve Problem
\[
\int_1^{\int_1^{\int_1^{\int_1^{.^{.^{.}}} 2x {d}x}2x {d}x}2x {d}x}2x {d}x \\
\]

\[
\begin{matrix}
 &    I = \int_1^{\int_1^{\int_1^{\int_1^{.^{.^{.}}} 2x {d}x}2x {d}x}2x {d}x}2x {d}x \\
 \implies & I = \int_1^{I} 2x{d}x \\
 \implies & I = I ^2 - 1 \\
 \implies & I = \frac {1 + \sqrt{5}} {2} \text { or } I = \frac {1 - \sqrt{5}} {2}  \\
\end{matrix}
\]

\section{MIT Integration Bee 2024 Finals Problem 2}
\href{https://www.youtube.com/watch?v=xdpKnkm2MT0}{MIT Integration Bee 2024 Finals Problem 2}



Problem  \[ \int_0^{\infty} \frac {\ln\left(2e^{x} - 1\right)} {e ^ {x} - 1} {d}x = \frac{\pi^2} {4}\]

Solve
\[
\begin{matrix}
        & I = \int_0^{\infty} \frac{\ln\left(2e^{x} - 1\right)} {e^{x}-1}{d}x \\
\implies & I \stackrel{u=e^{x}}{=} \int_1^{\infty} \frac{\ln\left(2u - 1\right)}{(u - 1)u} {d}u   \\
\implies & I \stackrel{t = \frac{1}{u}}{=======} \int_0^{1} \frac{\ln(2-t) - \ln(t)}{1-t} * t * t (-1) {d}t \\
\implies & I \stackrel {a = 1-t} {==============} \int_0^{1} \frac{\ln(1+a) - \ln(1-a)} {a} {d}a  \\
\implies & I = \int_0^1 \frac{\left({\displaystyle\sum_{k=1}^{\infty}\frac{(-1)^{(k-1)}a ^k}{k} + \displaystyle\sum_{k=1}^{\infty}\frac{-a^k}{k}} \right)}{a} {d}a \\
\implies & I = \int_0^1 \displaystyle\sum_{k=0}^{\infty} \frac{a^{2k}}{2k+1} {d}a \\
\implies & I = \displaystyle\sum_{k=0}^{\infty} \frac{a^{2k+1}}{(2k+1)^2} \Big|_{a=0}^{a=1} \\
\implies & I = \displaystyle\sum_{k=0}^{\infty} \frac{1} {(2k+1)^2} \\
\implies & \zeta(2) = \displaystyle\sum_{k=1}^{\infty}\frac{1}{k^2} = \frac{\pi^2}{6} \\
\implies & I = 2(\zeta(2) - \frac{\zeta(2)}{2^2}) = \frac{\pi^2}{4} \\
\end{matrix}
\]


\end{document}
